\Chapter{CONCLUSION}\label{sec:Conclusion}



\section{Synthèse des travaux}
Cette thèse a permis d'aborder des thèmes aussi variés que la segmentation sémantique de lésions, la représentation par graphe d'une image, sa classification et la conception de modèles de diagnostics auto-attentifs mettant l'accent sur l'interprétabilité du réseau. Avant de détailler les contributions propres à ces différents segments, rappelons que notre ambition initiale était de concevoir un modèle interprétable de classification des pathologies rétiniennes. Force est de constater que cette thèse n'y aboutit pas, mais au terme de ce travail cet apparent inachèvement est assumé. En effet, au lieu d'un modèle unique, universel et qui répondrait à cette ambition, nous avons proposé des expérimentations basées sur des méthodes très diverses voire transverses (CNN, GNN, ViT...), toutes plus ou moins capables de répondre au cahier des charges (reconnaissance automatique de pathologies rétiniennes) mais présentant toutes également des forces et faiblesses propres. Ce faisant, nous avons construit un chemin possible vers ce que le titre de cette thèse propose, c'est-à-dire un modèle interprétable de classification des  pathologies rétiniennes. Nous pouvons résumer les travaux menés en quelques phrases:
\begin{enumerate}
	\item Nous avons conçu un modèle de segmentation des lésions rétiniennes dans le \fundus{} et expérimenté l'entraînement sur des bases présentant différentes distributions d'annotations. Ce faisant, nous avons mis en évidence la capacité d'un modèle à s'adapter à ce qu'il identifie être l'origine d'une image. Cette identification peut être manipulée: nous avons introduit pour cela la notion de conversion adversariale de domaines.
	\item La classification via CNN des images ne s'appuie guère sur une segmentation explicite des lésions: celles-ci facilitent la convergence à l'entraînement mais n'ont que peu d'influence sur les performances à l'inférence. Afin de rendre explicite la dépendance entre lésions trouvées et classification de l'image (et donc d'estimer la qualité de la segmentation), nous avons proposé de créer une représentation par graphe de l'imagerie du \fundus{}, graphe que nous extrayons directement de la segmentation. À l'aide d'un réseau GNN, ce graphe peut être classifié. Si les performances ne s'approchent pas d'un CNN, un tel modèle présente des avantages en termes de poids, de modularité et d'interprétabilité.
	\item En s'éloignant de la segmentation sémantique, nous avons proposé diverses techniques permettant d'améliorer les performances d'un modèle de type Transformer. L'introduction de la notion de pas adaptatif a guidé le développement de l'Attention Concentrée afin de créer des cartes d'attributions plus précises et de plus haute résolution.
	\item Pour valider cette technique, nous avons mis en place un protocole permettant de prendre un compte l'opinion des médecins, chargés d'évaluer la qualité de diverses techniques d'attributions locales.
\end{enumerate}
Au cours du précédent chapitre, nous avons également présenté brièvement diverses expériences qui se sont servies des algorithmes développés. Cela nous a permis d'entrer dans le détail des limitations de chaque algorithme et par conséquent des perspectives d'améliorations futures.


